\documentclass[12pt,]{article}
\usepackage{lmodern}
\usepackage{amssymb,amsmath}
\usepackage{ifxetex,ifluatex}
\usepackage{fixltx2e} % provides \textsubscript
\ifnum 0\ifxetex 1\fi\ifluatex 1\fi=0 % if pdftex
  \usepackage[T1]{fontenc}
  \usepackage[utf8]{inputenc}
\else % if luatex or xelatex
  \ifxetex
    \usepackage{mathspec}
  \else
    \usepackage{fontspec}
  \fi
  \defaultfontfeatures{Ligatures=TeX,Scale=MatchLowercase}
    \setmainfont[]{Times New Roman}
\fi
% use upquote if available, for straight quotes in verbatim environments
\IfFileExists{upquote.sty}{\usepackage{upquote}}{}
% use microtype if available
\IfFileExists{microtype.sty}{%
\usepackage{microtype}
\UseMicrotypeSet[protrusion]{basicmath} % disable protrusion for tt fonts
}{}
\usepackage[margin=1in]{geometry}
\usepackage{hyperref}
\hypersetup{unicode=true,
            pdftitle={Seminar 1},
            pdfauthor={Lesnykh Kirill, BSC-153},
            pdfborder={0 0 0},
            breaklinks=true}
\urlstyle{same}  % don't use monospace font for urls
\usepackage{color}
\usepackage{fancyvrb}
\newcommand{\VerbBar}{|}
\newcommand{\VERB}{\Verb[commandchars=\\\{\}]}
\DefineVerbatimEnvironment{Highlighting}{Verbatim}{commandchars=\\\{\}}
% Add ',fontsize=\small' for more characters per line
\usepackage{framed}
\definecolor{shadecolor}{RGB}{248,248,248}
\newenvironment{Shaded}{\begin{snugshade}}{\end{snugshade}}
\newcommand{\KeywordTok}[1]{\textcolor[rgb]{0.13,0.29,0.53}{\textbf{#1}}}
\newcommand{\DataTypeTok}[1]{\textcolor[rgb]{0.13,0.29,0.53}{#1}}
\newcommand{\DecValTok}[1]{\textcolor[rgb]{0.00,0.00,0.81}{#1}}
\newcommand{\BaseNTok}[1]{\textcolor[rgb]{0.00,0.00,0.81}{#1}}
\newcommand{\FloatTok}[1]{\textcolor[rgb]{0.00,0.00,0.81}{#1}}
\newcommand{\ConstantTok}[1]{\textcolor[rgb]{0.00,0.00,0.00}{#1}}
\newcommand{\CharTok}[1]{\textcolor[rgb]{0.31,0.60,0.02}{#1}}
\newcommand{\SpecialCharTok}[1]{\textcolor[rgb]{0.00,0.00,0.00}{#1}}
\newcommand{\StringTok}[1]{\textcolor[rgb]{0.31,0.60,0.02}{#1}}
\newcommand{\VerbatimStringTok}[1]{\textcolor[rgb]{0.31,0.60,0.02}{#1}}
\newcommand{\SpecialStringTok}[1]{\textcolor[rgb]{0.31,0.60,0.02}{#1}}
\newcommand{\ImportTok}[1]{#1}
\newcommand{\CommentTok}[1]{\textcolor[rgb]{0.56,0.35,0.01}{\textit{#1}}}
\newcommand{\DocumentationTok}[1]{\textcolor[rgb]{0.56,0.35,0.01}{\textbf{\textit{#1}}}}
\newcommand{\AnnotationTok}[1]{\textcolor[rgb]{0.56,0.35,0.01}{\textbf{\textit{#1}}}}
\newcommand{\CommentVarTok}[1]{\textcolor[rgb]{0.56,0.35,0.01}{\textbf{\textit{#1}}}}
\newcommand{\OtherTok}[1]{\textcolor[rgb]{0.56,0.35,0.01}{#1}}
\newcommand{\FunctionTok}[1]{\textcolor[rgb]{0.00,0.00,0.00}{#1}}
\newcommand{\VariableTok}[1]{\textcolor[rgb]{0.00,0.00,0.00}{#1}}
\newcommand{\ControlFlowTok}[1]{\textcolor[rgb]{0.13,0.29,0.53}{\textbf{#1}}}
\newcommand{\OperatorTok}[1]{\textcolor[rgb]{0.81,0.36,0.00}{\textbf{#1}}}
\newcommand{\BuiltInTok}[1]{#1}
\newcommand{\ExtensionTok}[1]{#1}
\newcommand{\PreprocessorTok}[1]{\textcolor[rgb]{0.56,0.35,0.01}{\textit{#1}}}
\newcommand{\AttributeTok}[1]{\textcolor[rgb]{0.77,0.63,0.00}{#1}}
\newcommand{\RegionMarkerTok}[1]{#1}
\newcommand{\InformationTok}[1]{\textcolor[rgb]{0.56,0.35,0.01}{\textbf{\textit{#1}}}}
\newcommand{\WarningTok}[1]{\textcolor[rgb]{0.56,0.35,0.01}{\textbf{\textit{#1}}}}
\newcommand{\AlertTok}[1]{\textcolor[rgb]{0.94,0.16,0.16}{#1}}
\newcommand{\ErrorTok}[1]{\textcolor[rgb]{0.64,0.00,0.00}{\textbf{#1}}}
\newcommand{\NormalTok}[1]{#1}
\usepackage{graphicx,grffile}
\makeatletter
\def\maxwidth{\ifdim\Gin@nat@width>\linewidth\linewidth\else\Gin@nat@width\fi}
\def\maxheight{\ifdim\Gin@nat@height>\textheight\textheight\else\Gin@nat@height\fi}
\makeatother
% Scale images if necessary, so that they will not overflow the page
% margins by default, and it is still possible to overwrite the defaults
% using explicit options in \includegraphics[width, height, ...]{}
\setkeys{Gin}{width=\maxwidth,height=\maxheight,keepaspectratio}
\IfFileExists{parskip.sty}{%
\usepackage{parskip}
}{% else
\setlength{\parindent}{0pt}
\setlength{\parskip}{6pt plus 2pt minus 1pt}
}
\setlength{\emergencystretch}{3em}  % prevent overfull lines
\providecommand{\tightlist}{%
  \setlength{\itemsep}{0pt}\setlength{\parskip}{0pt}}
\setcounter{secnumdepth}{0}
% Redefines (sub)paragraphs to behave more like sections
\ifx\paragraph\undefined\else
\let\oldparagraph\paragraph
\renewcommand{\paragraph}[1]{\oldparagraph{#1}\mbox{}}
\fi
\ifx\subparagraph\undefined\else
\let\oldsubparagraph\subparagraph
\renewcommand{\subparagraph}[1]{\oldsubparagraph{#1}\mbox{}}
\fi

%%% Use protect on footnotes to avoid problems with footnotes in titles
\let\rmarkdownfootnote\footnote%
\def\footnote{\protect\rmarkdownfootnote}

%%% Change title format to be more compact
\usepackage{titling}

% Create subtitle command for use in maketitle
\newcommand{\subtitle}[1]{
  \posttitle{
    \begin{center}\large#1\end{center}
    }
}

\setlength{\droptitle}{-2em}

  \title{Seminar 1}
    \pretitle{\vspace{\droptitle}\centering\huge}
  \posttitle{\par}
    \author{Lesnykh Kirill, BSC-153}
    \preauthor{\centering\large\emph}
  \postauthor{\par}
      \predate{\centering\large\emph}
  \postdate{\par}
    \date{18 01 2019}

\newfontfamily{\cyrillicfonttt}{Times New Roman}
\usepackage[utf8]{inputenc}
\usepackage{fontspec}
\usepackage{polyglossia}
\setmainlanguage{russian}
\setotherlanguage{english}

\begin{document}
\maketitle

\begin{Shaded}
\begin{Highlighting}[]
\NormalTok{knitr}\OperatorTok{::}\NormalTok{opts_chunk}\OperatorTok{$}\KeywordTok{set}\NormalTok{(}\DataTypeTok{echo =} \OtherTok{TRUE}\NormalTok{)}
\end{Highlighting}
\end{Shaded}

\begin{Shaded}
\begin{Highlighting}[]
\KeywordTok{Sys.setlocale}\NormalTok{(}\DataTypeTok{locale =} \StringTok{"en_US.UTF-8"}\NormalTok{)}
\end{Highlighting}
\end{Shaded}

\begin{verbatim}
## [1] "en_US.UTF-8/en_US.UTF-8/en_US.UTF-8/C/en_US.UTF-8/C"
\end{verbatim}

Seminar Assignments

\begin{Shaded}
\begin{Highlighting}[]
\NormalTok{t <-}\StringTok{ }\KeywordTok{c}\NormalTok{(}\StringTok{"F"}\NormalTok{,}\StringTok{"D"}\NormalTok{,}\StringTok{"F"}\NormalTok{,}\StringTok{"I"}\NormalTok{,}\StringTok{"A"}\NormalTok{,}\StringTok{"F"}\NormalTok{,}\StringTok{"F"}\NormalTok{,}\StringTok{"D"}\NormalTok{,}\StringTok{"B"}\NormalTok{,}\StringTok{"SLO"}\NormalTok{,}\StringTok{"I"}\NormalTok{,}\StringTok{"F"}\NormalTok{,}\StringTok{"GB"}\NormalTok{,}\StringTok{"B"}\NormalTok{)}
\DecValTok{10}
\end{Highlighting}
\end{Shaded}

\begin{verbatim}
## [1] 10
\end{verbatim}

\begin{Shaded}
\begin{Highlighting}[]
\NormalTok{T <-}\StringTok{ }\KeywordTok{factor}\NormalTok{(t)}
\NormalTok{t}
\end{Highlighting}
\end{Shaded}

\begin{verbatim}
##  [1] "F"   "D"   "F"   "I"   "A"   "F"   "F"   "D"   "B"   "SLO" "I"  
## [12] "F"   "GB"  "B"
\end{verbatim}

\begin{Shaded}
\begin{Highlighting}[]
\KeywordTok{table}\NormalTok{(t)}\OperatorTok{/}\KeywordTok{length}\NormalTok{(t)}
\end{Highlighting}
\end{Shaded}

\begin{verbatim}
## t
##          A          B          D          F         GB          I 
## 0.07142857 0.14285714 0.14285714 0.35714286 0.07142857 0.14285714 
##        SLO 
## 0.07142857
\end{verbatim}

\begin{Shaded}
\begin{Highlighting}[]
\KeywordTok{cumsum}\NormalTok{(}\KeywordTok{table}\NormalTok{(t))}\OperatorTok{/}\KeywordTok{length}\NormalTok{(t)}
\end{Highlighting}
\end{Shaded}

\begin{verbatim}
##          A          B          D          F         GB          I 
## 0.07142857 0.21428571 0.35714286 0.71428571 0.78571429 0.92857143 
##        SLO 
## 1.00000000
\end{verbatim}

\begin{Shaded}
\begin{Highlighting}[]
\NormalTok{L <-}\StringTok{ }\KeywordTok{c}\NormalTok{(}\StringTok{"unsatisfactory"}\NormalTok{,}\StringTok{"poor"}\NormalTok{,}\StringTok{"average"}\NormalTok{,}\StringTok{"good"}\NormalTok{,}\StringTok{"excellent"}\NormalTok{)}

\NormalTok{L <-}\StringTok{ }\KeywordTok{c}\NormalTok{(}\StringTok{"Не удовлетворен"}\NormalTok{, }\StringTok{"Плохо"}\NormalTok{, }\StringTok{"Средне"}\NormalTok{, }\StringTok{"Хорошо"}\NormalTok{, }\StringTok{"Отлично"}\NormalTok{)}
\NormalTok{L}
\end{Highlighting}
\end{Shaded}

\begin{verbatim}
## [1] "Не удовлетворен" "Плохо"           "Средне"          "Хорошо"         
## [5] "Отлично"
\end{verbatim}

\begin{Shaded}
\begin{Highlighting}[]
\NormalTok{attributes <-}\StringTok{ }\KeywordTok{c}\NormalTok{(}\StringTok{"Income"}\NormalTok{, }\StringTok{"Age"}\NormalTok{, }\StringTok{"Number of Kids"}\NormalTok{, }\StringTok{"Education"}\NormalTok{)}
\NormalTok{Alex <-}\StringTok{ }\KeywordTok{c}\NormalTok{(}\DecValTok{2000}\NormalTok{, }\DecValTok{50}\NormalTok{, }\DecValTok{2}\NormalTok{, }\DecValTok{12}\NormalTok{)}
\NormalTok{Vlado <-}\StringTok{ }\KeywordTok{c}\NormalTok{(}\DecValTok{1000}\NormalTok{, }\DecValTok{25}\NormalTok{, }\DecValTok{1}\NormalTok{, }\DecValTok{16}\NormalTok{)}
\NormalTok{name <-}\StringTok{ }\KeywordTok{c}\NormalTok{(Alex, Vlado)}
\NormalTok{df <-}\StringTok{ }\KeywordTok{data.frame}\NormalTok{(Alex, Vlado, }\DataTypeTok{row.names =}\NormalTok{ attributes)}
\NormalTok{df <-}\StringTok{ }\KeywordTok{t}\NormalTok{(df)}
\NormalTok{df}
\end{Highlighting}
\end{Shaded}

\begin{verbatim}
##       Income Age Number of Kids Education
## Alex    2000  50              2        12
## Vlado   1000  25              1        16
\end{verbatim}

\begin{Shaded}
\begin{Highlighting}[]
\NormalTok{x <-}\StringTok{ }\KeywordTok{c}\NormalTok{(}\DecValTok{3}\NormalTok{,}\DecValTok{2}\NormalTok{,}\DecValTok{5}\NormalTok{)}
\NormalTok{x}
\end{Highlighting}
\end{Shaded}

\begin{verbatim}
## [1] 3 2 5
\end{verbatim}

The command creates the vector ``x''``, that contains following numeric
elements: 3, 2, 5.

\begin{Shaded}
\begin{Highlighting}[]
\NormalTok{A1 <-}\StringTok{ }\KeywordTok{matrix}\NormalTok{(}\KeywordTok{c}\NormalTok{(}\DecValTok{1}\NormalTok{,}\DecValTok{3}\NormalTok{,}\DecValTok{5}\NormalTok{,}\DecValTok{2}\NormalTok{,}\DecValTok{4}\NormalTok{,}\DecValTok{6}\NormalTok{),}\DecValTok{2}\NormalTok{,}\DecValTok{3}\NormalTok{)}
\NormalTok{A1}
\end{Highlighting}
\end{Shaded}

\begin{verbatim}
##      [,1] [,2] [,3]
## [1,]    1    5    4
## [2,]    3    2    6
\end{verbatim}

The command creates the matrix ``A1'' with 2 rows and 3 columns, that
contains the numeric elements of the vector c(1, 3, 5, 2, 4, 6). The
elements in the matrix are placed firstly for each row, than for each
column.

\begin{Shaded}
\begin{Highlighting}[]
\NormalTok{A <-}\StringTok{ }\KeywordTok{matrix}\NormalTok{(}\KeywordTok{c}\NormalTok{(}\DecValTok{1}\NormalTok{,}\DecValTok{4}\NormalTok{,}\DecValTok{2}\NormalTok{,}\DecValTok{5}\NormalTok{,}\DecValTok{3}\NormalTok{,}\DecValTok{6}\NormalTok{),}\DecValTok{2}\NormalTok{,}\DecValTok{3}\NormalTok{)}
\NormalTok{A}
\end{Highlighting}
\end{Shaded}

\begin{verbatim}
##      [,1] [,2] [,3]
## [1,]    1    2    3
## [2,]    4    5    6
\end{verbatim}

The command creates the matrix ``A'' with 2 rows and 3 columns, that
contains the numeric elements of the vector c(1, 4, 2, 5, 3, 6). The
elements in the matrix are placed firstly for each row, than for each
column. So the numbers in matrix are placed raising in amount firstly by
column, and than by row.

\begin{Shaded}
\begin{Highlighting}[]
\NormalTok{B <-}\StringTok{ }\KeywordTok{matrix}\NormalTok{(}\KeywordTok{c}\NormalTok{(}\DecValTok{4}\NormalTok{,}\DecValTok{1}\NormalTok{,}\DecValTok{5}\NormalTok{,}\DecValTok{2}\NormalTok{,}\DecValTok{6}\NormalTok{,}\DecValTok{3}\NormalTok{),}\DecValTok{2}\NormalTok{,}\DecValTok{3}\NormalTok{)}
\NormalTok{B}
\end{Highlighting}
\end{Shaded}

\begin{verbatim}
##      [,1] [,2] [,3]
## [1,]    4    5    6
## [2,]    1    2    3
\end{verbatim}

The command creates the matrix ``B'' with 2 rows and 3 columns, that
contains the numeric elements of the vector c(4, 1, 5, 2, 6, 3). The
elements in the matrix are placed firstly for each row, than for each
column.

\begin{Shaded}
\begin{Highlighting}[]
\NormalTok{X1<-A}\OperatorTok\KeywordTok{t}\NormalTok{(B)}
\NormalTok{X1}
\end{Highlighting}
\end{Shaded}

\begin{verbatim}
##      [,1] [,2]
## [1,]   32   14
## [2,]   77   32
\end{verbatim}

The command generates the correct multiplication of the matrixes ``A''
and ``B'' that equals to the object ``X1''. The command ``t()''
transponts the matrix ``B''. The element ``\%*\%" makes the correct
multiplication of the matrixes row by columns.

\begin{Shaded}
\begin{Highlighting}[]
\NormalTok{X2<-B}\OperatorTok\KeywordTok{t}\NormalTok{(A)}
\NormalTok{X2}
\end{Highlighting}
\end{Shaded}

\begin{verbatim}
##      [,1] [,2]
## [1,]   32   77
## [2,]   14   32
\end{verbatim}

The command generates the correct multiplication of the matrixes ``A''
and ``B'' that equals to the object ``X2''. The command ``t()''
transponts the matrix ``B''. The element ``\%*\%" makes the correct
multiplication of the matrixes row by columns.

\begin{Shaded}
\begin{Highlighting}[]
\NormalTok{X3<-}\KeywordTok{t}\NormalTok{(A)}\OperatorTok\NormalTok{B}
\NormalTok{X3}
\end{Highlighting}
\end{Shaded}

\begin{verbatim}
##      [,1] [,2] [,3]
## [1,]    8   13   18
## [2,]   13   20   27
## [3,]   18   27   36
\end{verbatim}

The command generates the correct multiplication of the matrixes ``A''
and ``B'' that equals to the object ``X3''. The command ``t()''
transponts the matrix ``A''. The element ``\%*\%" makes the correct
multiplication of the matrixes row by columns. Due to the another order
of the matrixes in the equation, transpoted matrix ``A'' stands before
matrix ``B'', the results would be different from the object ``X1''.

\begin{Shaded}
\begin{Highlighting}[]
\NormalTok{X4<-}\KeywordTok{t}\NormalTok{(B)}\OperatorTok\NormalTok{A}
\NormalTok{X4}
\end{Highlighting}
\end{Shaded}

\begin{verbatim}
##      [,1] [,2] [,3]
## [1,]    8   13   18
## [2,]   13   20   27
## [3,]   18   27   36
\end{verbatim}

The command generates the correct multiplication of the matrixes ``A''
and ``B'' that equals to the object ``X4''. The command ``t()''
transponts the matrix ``B''. The element ``\%*\%" makes the correct
multiplication of the matrixes row by columns. Due to the another order
of the matrixes in the equation, transpoted matrix ``B'' stands before
matrix ``A'', the results would be different from the object ``X2''.

\begin{Shaded}
\begin{Highlighting}[]
\NormalTok{d<-}\KeywordTok{diag}\NormalTok{(A}\OperatorTok\KeywordTok{t}\NormalTok{(B))}
\NormalTok{d}
\end{Highlighting}
\end{Shaded}

\begin{verbatim}
## [1] 32 32
\end{verbatim}

The command generates the object ``b'' which contains the results of the
summary of the A and transpotted B and after that (due to the dial()
function) deletes all other numbers exceot for the ones from the
diagonal. So, in the output we have only the numbers from the diagonal
of the generated matrix.

\begin{Shaded}
\begin{Highlighting}[]
\NormalTok{s<-}\KeywordTok{sum}\NormalTok{(}\KeywordTok{diag}\NormalTok{(A}\OperatorTok\KeywordTok{t}\NormalTok{(B)))}
\NormalTok{s}
\end{Highlighting}
\end{Shaded}

\begin{verbatim}
## [1] 64
\end{verbatim}

The command generates the object ``s'' which contains the sum of the
results of the previouos line: it generates the diagonal numbers of the
sum of the matrixes A and transpoted B. So, the generated number in the
output is the sum of the diagonal numbers of the sum of the matrixes A
and transpoted B.

\begin{Shaded}
\begin{Highlighting}[]
\NormalTok{Y<-}\KeywordTok{solve}\NormalTok{(A}\OperatorTok\KeywordTok{t}\NormalTok{(B))}
\NormalTok{A}
\end{Highlighting}
\end{Shaded}

\begin{verbatim}
##      [,1] [,2] [,3]
## [1,]    1    2    3
## [2,]    4    5    6
\end{verbatim}

\begin{Shaded}
\begin{Highlighting}[]
\KeywordTok{t}\NormalTok{(B)}
\end{Highlighting}
\end{Shaded}

\begin{verbatim}
##      [,1] [,2]
## [1,]    4    1
## [2,]    5    2
## [3,]    6    3
\end{verbatim}

\begin{Shaded}
\begin{Highlighting}[]
\NormalTok{Y}
\end{Highlighting}
\end{Shaded}

\begin{verbatim}
##            [,1]       [,2]
## [1,] -0.5925926  0.2592593
## [2,]  1.4259259 -0.5925926
\end{verbatim}

The command generates the object ``Y'' which generates the matrix. If
the matrix ``X1'', the result of the correct multipication of the A and
transpoted B, would be multiplied by the matrix ``Y'', the result would
be the identity matrix. That is what the function ``solve'' does: it
gives the matrix, multiplication by which of the given matrix will equal
unity matrix.


\end{document}
